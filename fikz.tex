\documentclass{article}
\usepackage[margin=1.2in]{geometry}
\usepackage{tikz, fikz, hyperref, setspace, xcolor}

\renewcommand{\textbackslash}{\char`\\}
\def\leftbrace{\char`\{}
\def\rightbrace{\char`\}}

\definecolor{cmd}{HTML}{214a87}
\def\cmd#1{\texttt{\color{cmd}\textbackslash#1}}
\def\arg#1{{\color{cmd}\leftbrace}{\color{black}#1}{\color{cmd}\rightbrace}}
\definecolor{opt}{HTML}{c4a100}
\def\optarg[#1]{{\color{opt}[#1]}}

\title{\vspace{-7mm}The \textsf{fikz} package\footnote{This file has version number v1.1, last revised on~\today.}}
\author{Benjamin Bernard}

\linespread{1.15}

\begin{document}
\maketitle

\vspace{-1mm}
\begin{abstract}
This \LaTeX\ package changes the basic overlay behavior of \texttt{tikz} macros from the \cmd{only} functionality to the \cmd{onslide} functionality, thereby solving the ``jumping image'' issue.
\end{abstract}

\section{Introduction}

The \texttt{beamer} document class introduces overlay functionality to gradually reveal information on frames. The standard overlay functionality added to the \texttt{tikz} macros is the \cmd{only} functionality, which reserves space for the generated path only on the slides, on which the paths are visible. This causes gradually revealed \texttt{tikz} pictures to ``jump around'' from slide to slide if the picture's bounding box changes. This package changes the default overlay functionality of the \texttt{tikz} macros to the \cmd{onslide} functionality, which reserves space for the generated path on all frames of the slide. Because I did not find a neat solution on \href{https://tex.stackexchange.com}{\texttt{tex.stackexchange.com}}, I posted this problem at \url{https://tex.stackexchange.com/questions/599624/replace-tikzs-standard-overlay-specification-from-only-to-onslide},\linebreak where I also first posted the solution that gave rise to this package.

%The problem of jumping figures 

The package solves two additional overlay-related compatibility issues between \texttt{beamer} and non-\texttt{beamer} classes. First, the package suppresses any overlay specifications provided to \texttt{tikz} macros in non-\texttt{beamer} document classes. This allows the same code of a \texttt{tikzfigure} picture to be used in the paper and a presentation, or in an assignment and the lecture slides. Finally, \texttt{tikz} provides a neat externalization library, with which drawn \texttt{tikzfigures} can be exported to a separate document and imported in future compilations. Some minor workaround is necessary to make this feature integrate seamlessly with \texttt{beamer}.

\section{Usage}

The main functionality is added by loading the package with \cmd{usepackage\arg{fikz}} in the preamble. Load the package with the option \texttt{external} if you also wish to add the support for externalization. If the current document is of the \texttt{beamer} document class, loading the package will change the default overlay behavior of the commands \cmd{draw}, \cmd{fill}, \cmd{filldraw}, \cmd{clip}, and \cmd{node} from the \cmd{only} functionality to the \cmd{onslide} functionality. If the current document is of any other document class, overlay instructions \texttt{<>} are ignored for all the above commands.

If the package is included in the definition of a document class that builds on \texttt{beamer}, the package needs to be loaded with \cmd{RequirePackage\optarg[beamer]\arg{fikz}}. If the package is included in the definition of any other document class, it needs to be loaded with \cmd{RequirePackage\optarg[doc]\arg{fikz}}.

The only command provided by the package is the \cmd{externalize\arg{<name>}}. Similar to \texttt{tikz}' native \cmd{tikzsetnextfilename}, call \cmd{externalize} before any \texttt{tikzpicture} to export the figure to a separate pdf. The name of the document is \texttt{<name>-<overlay>.pdf} in any non-handout \texttt{beamer} mode or \texttt{<name>.pdf} in \texttt{beamer}'s handout mode or in any other document class. Simply by assigning different names to figures from different overlays, the externalization behaves as expected. 

%\option{external} enables support for Tikz's externalization library. Write \cmd{externalize\arg{<name>}} before any \texttt{tikzpicture} or any environment from this package to export the figures to a separate pdf. The name of the document is \texttt{<name>-<overlay>.pdf} in any non-handout beamer mode or \texttt{<name>.pdf} in beamer's handout mode or in any other document class. For the environments from this package, you can alternatively set the key \texttt{external=<name>} at the environment level. Externalization is, of course, very slow during the first compilation. However, figures will not have to be redrawn in any future compilations (unless there are changes to the figures). If you specify the same \texttt{<name>} for two or more figures, only the first one will be drawn and exported, the others will be imported.

\section{Implementation}

The \texttt{tikz} package redefines the commands \cmd{draw}, \cmd{fill}, \cmd{filldraw}, \cmd{clip}, and \cmd{node} at the beginning of every \verb|tikzpicture| environment so that their functionality is preserved even if the user defines their own macro with the same name. This package taps into this mechanism by parsing the overlay specification separately in this redefinition and then calling \cmd{onslide<>{<original definition>}} in the \texttt{beamer} document class and ignoring the overlay specification in other document classes.

The active document class is determined with \cmd{@ifclassloaded}. Since this macro is not available in the definition of a document class, the package needs to be loaded with options \texttt{beamer} or \texttt{doc} in this case, depending on which of the two functionalities is desired.

\end{document}
